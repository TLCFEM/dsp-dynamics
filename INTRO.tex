\section{Introduction}
Within the typical dynamic analysis workflow, various types of spurious responses stemming from different parts have been identified by researchers. To name a few,
\begin{enumerate}
\item error contained in the collected/applied input \citep{Shing1987},
\item in the case of damped system, the specific damping model used \citep{Hall2006,Jehel2014,Chopra2015},
\item properties of the numerical time integration methods (associated with the spurious root) \citep{Hulbert1994}.
\end{enumerate}
Each of the processes must be taken care of with great caution in order to obtain reasonable and reliable numerical results.

In the context of the finite element response history analysis, seismic action can be introduced into the system via either force or deformation. For the former, acceleration record is converted to inertial force with the assist of mass and then applied to target degrees of freedom. For the latter, displacement record can be applied to supports to shake the structure as if it sits on a shaking table.
The typical sampling rate of ground motion seismograms ranges from \SI{50}{\hertz} to \SI{200}{\hertz}. This corresponds to a sampling period $T_s$ between \SI{5}{\milli\second} and \SI{20}{\milli\second}. Often such a time step size is not sufficient for response history analysis, either linear or non-linear, due to accuracy and convergence issues \citep[see, e.g.,][]{Chang2011,Rossi2014}. Ideally, the continuous--time version needs to be reconstructed from the discrete--time version of input seismogram in order to allow a smaller time step size to be used for simulation. However, it is obvious that the exact reconstruction is not achievable.

It is then a convention to perform linear interpolation between two adjacent discrete samples for the case that time step size (of numerical analysis) is different from sampling interval (of input seismogram). For example, ABAQUS \citep[see][\S34.1.2]{ABAQUS2014} provides functionalities to accept load with its amplitude defined in tabular form, linear interpolation is automatically performed with optional smoothing within a small range ($\pm\tau$) around sharp turning points. Essentially, the linear interpolation is equivalent to a low--pass finite impulse response (FIR) filter with a Bartlett (triangular) window.

Although the numerical error introduced by time integration methods and its impact on both linear and non-linear systems are well studied \citep[see, e.g.,][]{Chang2003,Chang2005}, the side effect of linear interpolation on numerical results is not well discussed. In this work, we identify one type of spurious responses caused by linear interpolation of input force/deformation. In order to reveal it quantitatively, the analytical expressions for damping/inertial forces of linear systems under upsampled external loads in the frequency domain are derived using signal processing methods. It is then further elaborated with the numerical simulation of a simple SDOF oscillator. Remedies and recommendations to ease the issue are then discussed. To showcase its impact on practical structures, a constrained cantilever is analysed under seismic loading. Based on the discussions, a revised analysis workflow is proposed for response history analysis.