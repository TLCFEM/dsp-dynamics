\section{Introduction}
Within a typical dynamic analysis workflow, various types of spurious responses stemming from different parts have been identified by researchers. To name a few,
\begin{enumerate}
\item error contained in the collected/applied input \citep{Shing1987},
\item in the case of damped system, the specific damping model used \citep{Hall2006,Jehel2014,Chopra2015},
\item properties of the numerical time integration methods (associated with the spurious root) \citep{Hulbert1994}.
\end{enumerate}
Each of the processes must be taken care of with great caution in order to obtain reasonable and reliable numerical results.

In the context of the finite element response history analysis, seismic actions can be introduced into the system via either force or deformation. For the former, an acceleration record is converted to an inertial force with the assist of mass matrix and then applied to the target degrees of freedom. For the latter, a displacement record can be applied to supports to shake the structure as if it sits on a shaking table.
The typical sampling rate of ground motion seismograms ranges from \SI{50}{\hertz} to \SI{200}{\hertz}. This corresponds to a sampling period $T_s$ between \SI{5}{\milli\second} and \SI{20}{\milli\second}. Such a time step size is often not sufficient for either linear or non-linear response history analysis due to accuracy and convergence issues \citep[see, e.g.,][]{Chang2011,Rossi2014}. Ideally, the continuous--time version of the input seismogram needs to be reconstructed from the discrete--time version in order to allow a smaller time step size to be used for simulation. However, the exact reconstruction is not achievable.

It is a convention to perform linear interpolation between two adjacent discrete samples in the case that the time step size (of numerical analysis) is different from the sampling interval (of input seismogram). For example, ABAQUS \citep[see][\S34.1.2]{ABAQUS2014} provides functionalities to define loads in a tabular form, a linear interpolation is automatically performed with optional smoothing within a small range ($\pm\tau$) around sharp turning points. Essentially, the linear interpolation is equivalent to a low--pass finite impulse response (FIR) filter with a Bartlett (triangular) window. Although the numerical error introduced by time integration methods and its impact on both linear and non-linear systems are well studied \citep[see, e.g.,][]{Chang2003,Chang2005}, the side effect of such a linear interpolation procedure on numerical results is rarely discussed.

The present study investigates a particular type of spurious responses resulting from the linear interpolation of input force/deformation. These responses are analysed analytically, providing a clear evidence that the isolated error is entirely caused by interpolation and not by any other factors. To quantify this error, analytical expressions for damping and inertial forces of linear systems under interpolated or upsampled external loads are derived using signal processing methods. The results are then further elaborated upon using numerical simulations. To showcase the impact of this phenomenon on practical structures, a constrained cantilever under seismic loading is analysed. Remedies and recommendations to address this issue are also discussed. Finally, based on the discussions, a revised analysis workflow is proposed to improve the quality and reliability of numerical results in response history analysis.